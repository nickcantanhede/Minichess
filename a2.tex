\documentclass[fontsize=11pt]{article}
\usepackage{amsfonts}
\usepackage{amsmath}
\usepackage{amsthm}
\usepackage[utf8]{inputenc}
\usepackage[margin=0.75in]{geometry}

\title{CSC111 2026 Assignment 2}
\author{Nicolas Miranda Cantanhede}
\date{\today}

% Some useful LaTeX commands. You are free to use these or not, and also add your own.
\newcommand{\N}{\mathbb{N}}
\newcommand{\Z}{\mathbb{Z}}
\newcommand{\R}{\mathbb{R}}
\newcommand{\cO}{\mathcal{O}}
\newcommand{\floor}[1]{\left\lfloor #1 \right\rfloor}

\begin{document}
\maketitle

\section*{Part 3: The Combinatorial Explosion}

\begin{enumerate}

\item[1.] A player has AT MOST 8 pieces on minichess. They are: \textbf{1 king, 1 queen, 2 rooks and 4 pawns}.
\begin{itemize}
    \item A king has at most \textbf{8 possible moves} as it can move to all 8 neighbour squares.
    \item A rook has at most \textbf{6 possible moves} as it can move 3 squares along its row and 3 along its column
    \item In a similar manner to the rook, the queen has 6 moves plus at most 5 diagonal moves. This sums up to a total of \textbf{11 possible moves}.
    \item Finally, the pawn has at most \textbf{3 moves}. 1 for going forward and two for diagonal captures (left and right).
\end{itemize}

Hence, combining all pieces possible moves, we get:
$$8 + 2 \cdot 6 + 11 + 4 \cdot 3 = \boxed{43 \,\, \text{possible moves}}$$


\item[2.] At depth 0, we only have one possible node in our game tree. \\

At depth 1, each root has at most 43 children (from Q1). Following this same pattern, at depth 2, each of those 43 children have at most 43 children. \\

Since we have a maximum of 50 moves, we can go up to a depth of 50. Hence the upper bound for the total number of nodes is:
$$43^0 + 43 + 43^2 + 43^3 + 43^4 + \cdots+43^{50} = \dfrac{43^{51} - 1}{43 - 1} = \boxed{\dfrac{43^{51} -1}{42} \,\, \text{nodes}}$$

\end{enumerate}
\end{document}
